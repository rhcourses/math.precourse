\begin{taskframes}{3.105}

\begin{taskframe}
    \begin{mathbox}{Funktion und Ableitungen}
        \begin{align*}
            f(x) &= \frac{1}{1+x^2} & f''(x) &= 2 f^2(x) (4x^2 f(x) - 1) \\
            f'(x) &= -2x \cdot f(x) 
        \end{align*}
    \end{mathbox}

    \begin{tcbraster}[raster columns=2, raster equal height]
        \begin{textbox}{Nullstellen, Lücken}
            Keine!
        \end{textbox}
        \begin{mathbox}{Verhalten für $x \to \pm \infty$}
            \begin{align*}
                \lim_{x \to \pm \infty} f(x) &= 0
            \end{align*}
        \end{mathbox}
    \end{tcbraster}
    \begin{mathbox}{Definitions- und Wertebereich}
        \begin{align*}
            D_f &= \mathbb{R} & W_f &= (0, 1]
        \end{align*}
    \end{mathbox}
\end{taskframe}

\begin{taskframe}
    \begin{mathbox}{Funktion und Ableitungen}
        \begin{align*}
            f(x) &= \frac{1}{1+x^2} & f''(x) &= 2 f^2(x) (4x^2 f(x) - 1) \\
            f'(x) &= -2x \cdot f(x)
        \end{align*}
    \end{mathbox}
    \begin{mathbox}{Nullstellen von $f'$ und $f''$}
        \begin{align*}
            f'(x) &= 0 & \Leftrightarrow && -2x &= 0 & \Leftrightarrow && x &= 0 \\
            f''(x) &= 0 & \Leftrightarrow && 4x^2 f(x) - 1 &= 0 \\
            && \Leftrightarrow && 4x^2 \frac{1}{1+x^2} - 1 &= 0 \\
            && \Leftrightarrow && 4x^2 -1 -x^2 &= 0 \\
            && \Leftrightarrow && 3x^2 -1 &= 0
            & \Leftrightarrow && x &= \pm\frac{1}{\sqrt{3}}
        \end{align*}
    \end{mathbox}
\end{taskframe}

\begin{taskframe}
    \begin{mathbox}{Funktion und Ableitungen}
        \begin{align*}
            f(x) &= \frac{1}{1+x^2} & f''(x) &= 2 f^2(x) (4x^2 f(x) - 1) \\
            f'(x) &= -2x \cdot f(x)
        \end{align*}
    \end{mathbox}
    \begin{mathbox}{Nullstellen von $f'$ und $f''$}
        \begin{align*}
            f'(x) &= 0 & &\Leftrightarrow & x &= 0 & && &&
            f''(x) &= 0 & &\Leftrightarrow & x &= \pm\frac{1}{\sqrt{3}}
        \end{align*}
    \end{mathbox}
    \begin{mathbox}{Extremstellen}
        \begin{align*}
            f''(0) &= -2 \text{, also ist $(0,f(0)) = (0,1)$ ein Maximum.}
        \end{align*}
    \end{mathbox}
\end{taskframe}

\begin{taskframe}
    \begin{mathbox}{Funktion und Ableitungen}
        \begin{align*}
            f(x) &= \frac{1}{1+x^2} & f''(x) &= 2 f^2(x) (4x^2 f(x) - 1) \\
            f'(x) &= -2x \cdot f(x)
        \end{align*}
    \end{mathbox}
    \begin{mathbox}{Nullstellen von $f'$ und $f''$}
        \begin{align*}
            f'(x) &= 0 & &\Leftrightarrow & x &= 0 & && &&
            f''(x) &= 0 & &\Leftrightarrow & x &= \pm\frac{1}{\sqrt{3}}
        \end{align*}
    \end{mathbox}
    \begin{mathbox}{Wendestellen}
        \begin{align*}
            &(-\frac{1}{\sqrt{3}}, \frac{3}{4}) & &\text{und} & &(\frac{1}{\sqrt{3}}, \frac{3}{4})
        \end{align*}
    \end{mathbox}
\end{taskframe}

\end{taskframes}
