\begin{taskframes}{3.108}

\begin{taskframe}
    \begin{mathbox}{Funktion und Ableitungen}
        \begin{align*}
            f(x) &= \frac{x^2}{1+x} & f'(x) &= \frac{1 - x^2}{(1+x^2)^2} & f''(x) &= \frac{2x(x^2-3)}{(1+x^2)^3}
        \end{align*}
    \end{mathbox}

    \begin{tcbraster}[raster columns=2, raster equal height]
        \begin{textbox}{Nullstellen, Lücken}
            Nullstelle bei $x = 0$
        \end{textbox}
        \begin{mathbox}{Verhalten für $x \to \pm \infty$}
            \begin{align*}
                &\lim_{x \to -\infty} f(x) = -0 \\
                &\lim_{x \to +\infty} f(x) = +0
            \end{align*}
        \end{mathbox}
    \end{tcbraster}
    \begin{mathbox}{Definitions- und Wertebereich}
        \begin{align*}
            D_f &= \mathbb{R} & W_f &= \left[-\frac{1}{2}, +\frac{1}{2}\right]
        \end{align*}
    \end{mathbox}
\end{taskframe}

\begin{taskframe}
    \begin{mathbox}{Funktion und Ableitungen}
        \begin{align*}
            f(x) &= \frac{x^2}{1+x} & f'(x) &= \frac{1 - x^2}{(1+x^2)^2} & f''(x) &= \frac{2x(x^2-3)}{(1+x^2)^3}
        \end{align*}
    \end{mathbox}
    \begin{mathbox}{Nullstellen von $f'$ und $f''$}
        \begin{align*}
            f'(x) &= 0 & \Leftrightarrow && x^2 &= 1 & \Leftrightarrow && x &= \pm 1 \\
            f''(x) &= 0 & \Leftrightarrow && 2x&= 0 \vee x^2-3 = 0 &\Leftrightarrow && x &\in \{0,-\sqrt{3}, \sqrt{3}\}
        \end{align*}
    \end{mathbox}
\end{taskframe}

\begin{taskframe}
    \begin{mathbox}{Funktion und Ableitungen}
        \begin{align*}
            f(x) &= \frac{x^2}{1+x} & f'(x) &= \frac{1 - x^2}{(1+x^2)^2} & f''(x) &= \frac{2x(x^2-3)}{(1+x^2)^3}
        \end{align*}
    \end{mathbox}
    \begin{mathbox}{Nullstellen von $f'$ und $f''$}
        \begin{align*}
            f'(x) &= 0 \Leftrightarrow x = \pm 1 & &&
            f''(x) &= 0 \Leftrightarrow x \in \{0,-\sqrt{3}, \sqrt{3}\}
        \end{align*}
    \end{mathbox}
    \begin{mathbox}{Extremstellen}
        \begin{align*}
            f''(-1) &= \frac{1}{2} \text{, also ist $(0,f(0)) = (0,-\frac{1}{2})$ ein Minimum.} \\
            f''(1) &= -\frac{1}{2} \text{, also ist $(0,f(0)) = (0,\frac{1}{2})$ ein Maximum.}
        \end{align*}
    \end{mathbox}
\end{taskframe}

\begin{taskframe}
    \begin{mathbox}{Funktion und Ableitungen}
        \begin{align*}
            f(x) &= \frac{x^2}{1+x} & f'(x) &= \frac{1 - x^2}{(1+x^2)^2} & f''(x) &= \frac{2x(x^2-3)}{(1+x^2)^3}
        \end{align*}
    \end{mathbox}
    \begin{mathbox}{Nullstellen von $f'$ und $f''$}
        \begin{align*}
            f'(x) &= 0 \Leftrightarrow x = \pm 1 & &&
            f''(x) &= 0 \Leftrightarrow x \in \{0,-\sqrt{3}, \sqrt{3}\}
        \end{align*}
    \end{mathbox}
    \begin{mathbox}{Wendestellen}
        \begin{align*}
            &(-\sqrt{3}, -\frac{\sqrt{3}}{4}) & &(0,0) & &(\sqrt{3}, \frac{\sqrt{3}}{4})
        \end{align*}
    \end{mathbox}
\end{taskframe}

\end{taskframes}
